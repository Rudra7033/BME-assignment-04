\documentclass{article}
% Comment the following line to NOT allow the usage of umlauts
\usepackage[utf8]{inputenc}
% Uncomment the following line to allow the usage of graphics (.png, .jpg)
%\usepackage{graphicx}

% Start the document
\begin{document}

% Create a new 1st level heading
\section{Disruptive Innovations in Healthcare}

\subsection{INTRODUCTION}
The healthcare industry is no stranger to innovation. Every day, new medicines,
medical equipment, and healthcare management approaches are introduced.
Disruptive innovations in healthcare, on the other hand, were significantly less
common until lately. What exactly is disruptive innovation, and how does it
affect the healthcare industry? Disruptive innovations are those that bring
about significant change and frequently result in the emergence of new industry
leaders. They upend the status quo to such an extent that it has a knock-on
effect throughout the sector.

sruptive innovation is the process of transforming expensive or highly sophis-
ticated products or services that were previously only available to a high-end
or more skilled part of the population into products or services that are more
affordable and accessible to a wider audience.

Take the automobile for example. In the late 19th century, cars were
not a disruptive innovation. They were luxury items occupying the high-end
of the transportation market. Most people continued using more affordable
horse-drawn vehicles to travel.

\subsection{AI and machine learning}

AI can compile and analyse survey results
thanks to its natural language processing skills. AI will most likely become more
widely used as a means of lowering healthcare expenses and allowing doctors
and staff to focus on patient care. The concerns around database management
and patient privacy must be understood by healthcare leaders.
AI applications can manage patient intake and scheduling as well as billing.
Chatbots answer patient questions. With natural language processing ca-
pabilities, AI can collate and analyze survey responses. AI will probably
increase in use as a way to bring down healthcare costs and let doctors and
staff focus on patient care. Healthcare leaders must be knowledgeable about
the issues surrounding database management and patient privacy


\subsection{Blockchain}
Blockchain is a database technology that stores data and links it in a secure
and usable way using encryption and other security features. Many elements
of healthcare are made easier as a result of this invention, including patient
records, supply and distribution, and research. With blockchain applications,
IT entrepreneurs have joined the healthcare sector, changing how providers
handle medical data.
Blockchain is a database technology that stores data and links it in a secure
and usable way using encryption and other security features. Many elements
of healthcare are made easier as a result of this invention, including patient
records, supply and distribution, and research. With blockchain applications,
IT entrepreneurs have joined the healthcare sector, changing how providers
handle medical data.

\subsection{Telemedicine}
COVID-19 has unquestionably accelerated telemedicine delivery, and experts
agree that telemedicine is here to stay. It works, doctors will get paid for
telemedicine consultations, and many patients prefer it. Telemedicine, on the
other hand, is heavily reliant on internet connectivity, and some parts of the
United States still lack it.
\subsection{Consumer-centered care}
Consumer-centered care is an example of a disruptive innovation in healthcare.
The patient-healthcare-provider interaction has changed dramatically as health-
care has become more consumerized. The mix of technology and public policy
has altered how patients obtain healthcare and engage with their physicians in
this area.
Data security in electronic health records, billing transparency, and access to
medical records are all part of a dramatic shift in healthcare that guarantees
patients have all the information they need to make informed decisions about
their care. According to the Centers for Medicare Medicaid Services, hospitals
must make their costs more transparent by early 2021. (CMS). Other future
revisions include the use of internet pricing tools that allow patients to see their
out-of-pocket expenses.

Since the passage of the Affordable Treatment Act, electronic health records
(EHRs) have become an increasingly important aspect of patient care. The huge
volume of EHR data, on the other hand, may be utilised to perform research,
improve care, construct AI applications, and create new economic prospects in
addition to patient health records. As a result, healthcare practitioners must
be aware of the concerns about EHR securit.


% Uncomment the following two lines if you want to have a bibliography
%\bibliographystyle{alpha}
%\bibliography{document}

\end{document}
